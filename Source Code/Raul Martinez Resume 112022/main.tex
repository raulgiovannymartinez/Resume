%%%%%%%%%%%%%%%%%%%%%%%%%%%%%%%%%
%%% CV Jakša Tomović 06/2019 %%%%%%
%%%%%%%%%%%%%%%%%%%%%%%%%%%%%%%%%

\PassOptionsToPackage{dvipsnames}{xcolor}
\documentclass[10pt,a4paper]{altacv}

%Layout
\geometry{left=0.5cm,right=8cm,marginparwidth=7cm,marginparsep=0.5cm,top=0.50cm,bottom=0.50cm,footskip=2\baselineskip}

%Packages
\usepackage[utf8]{inputenc}
\usepackage[T1]{fontenc}
\usepackage[default]{lato}
\usepackage{hyperref}

%Colors

\definecolor{accent}{HTML}{000e17}
\definecolor{heading}{HTML}{000e17}
\definecolor{emphasis}{HTML}{696969}
\definecolor{body}{HTML}{01415f}

\colorlet{heading}{heading}
\colorlet{accent}{accent}
\colorlet{emphasis}{emphasis}
\colorlet{body}{body}

\renewcommand{\itemmarker}{{\small\textbullet}}
\renewcommand{\ratingmarker}{\faCircle}

%

\begin{document}
\name{Raul G. Martinez}
\tagline{}
%\photo{2.5cm}{BrunoAlves}
\personalinfo{
    %\printinfo{\faBirthdayCake}{Aug 10th, 1994}
    \email{gio.mtz3@gmail.com}
    \phone{+1 619-734-6231}
%    \mailaddress{SR Njemačke 6, 10000}
    \location{San Diego, CA}
    \linkedin{raul-giovanny-martinez-688b1bb4}
    %\homepage{x}
    \github{github.com/raulgiovannymartinez}
}

%

\begin{fullwidth}
\makecvheader
\end{fullwidth}

%

%%%%%%%%%%%%%%%%%%%%%%%%%%%%%%% Experience %%%%%%%%%%%%%%%%%%%%%%%%%%%%%%%

\cvsection[page1sidebar]{Work Experience}

%%%%%%%%%%%%%
%Experience 0
%%%%%%%%%%%%%

\cvevent{Data Engineer III, Thermo Fisher Scientific}{}{June 2021 -- Present}{Carlsbad, CA}

\begin{itemize}
    % \setlength{\itemindent}{0.5em}
    \item   \small{Designed AWS Cloud Architectures, Batch Processing Pipelines, Data QC, and Custom ETL Code for orchestrating big data applications in qPCR COVID-19 clinical results.}
    % \setlength{\itemindent}{0.5em}
    \item   \small{Supported data science efforts for tracking and monitoring pandemic related topics through dashboards and analysis for external customer's specific data and internal product investigations.}
\end{itemize}

\medskip

%%%%%%%%%%%%%
%Experience 1
%%%%%%%%%%%%%

\cvevent{Data Engineer II, Illumina}{}{July 2018 -- June 2021}{San Diego, CA}

\begin{itemize}
    % \setlength{\itemindent}{0.5em}
    \item   \small{Developed Data Integration, ETL, and Data Modeling solutions for instrument's hardware tests and sequencing data across all NGS platforms.}
    % \setlength{\itemindent}{0.5em}
    % \item   \small{Supported relational database systems (SQL server) for storage, transformation, and retrieval.}
    % \setlength{\itemindent}{0.5em}
    \item   \small{Enabled ad-hoc analytics, reporting, and insight generation for key metrics and transformed outputs for the many NGS core components such as optics, micro-fluidics, firmware/software, chemistry. Along stages of product development and manufacturing.}
    % \setlength{\itemindent}{0.5em}
    %\item   \small{Created data pipelines (R, Python, MATLAB) for aggregation, analysis, and visualization (Dashboarding) of NGS instruments data including primary/secondary metrics, engineering tests, reports, and diagnostic logs.}
    % \setlength{\itemindent}{0.5em}
    % \item   \small{Innovated algorithms for image processing, manipulation, and transformation using python libraries (i.e. NumPy, PIL).}
\end{itemize}

\medskip

%%%%%%%%%%%%%
%Experience 1.5
%%%%%%%%%%%%%

% \cvevent{Integration Research Associate I-II, Illumina}{}{July 2018 -- July 2020}{San Diego, CA}

% \begin{itemize}
%     \setlength{\itemindent}{0.5em}
%     \item   \small{Collaborated cross-functionally to move forward new integration testing plans (optics,hardware,software, and chemistry components) and Design of Experiments (DOEs) for critical parameter analysis.}
%     \setlength{\itemindent}{0.5em}
%     \item   \small{Implemented data collection strategies with QC targets for NGS instruments in order to monitor optical performance across the fleet over time.}
% \end{itemize}

% \medskip

%%%%%%%%%%%%%
%Experience 3
%%%%%%%%%%%%%

\cvevent{Pharma-Technical Development Research, Genentech/Roche}{}{February 2018 -- July 2018}{South San Francisco, CA}

\begin{itemize}
    % \setlength{\itemindent}{0.5em}
    \item   \small{Collaborated in process development/purification in viral clearance studies (virus removal, inactivation) for the production of therapeutic monoclonal antibodies.}
    % \setlength{\itemindent}{0.5em}
    \item   \small{Investigated and optimized virus titer infectivity assays by implementing design of experiments (using SAS JMP), inferential statistics, and regression analysis.}
\end{itemize}

\medskip

%%%%%%%%%%%%%
%Experience 4
%%%%%%%%%%%%%

\cvevent{Software Test Engineer, Dotmatics}{}{December 2017 -- February 2018}{San Diego, CA}

\begin{itemize}
    % \setlength{\itemindent}{0.5em}
    \item   \small{Collaborated with Software Developers and Application Scientists to troubleshoot customer-specific issues and enhancements for the many scientific/biological informatics product suite offerings by testing software front-end features and components, UX/UI, software architecture, and Oracle relational databases.}
\end{itemize}

\medskip

%%%%%%%%%%%%%
%Experience 5
%%%%%%%%%%%%%

\cvevent{Data Analyst, Retrovirox}{}{May 2016 -- July 2017}{San Diego, CA}


\begin{itemize}
%   \setlength{\itemindent}{0.5em}
    \item   \small{Implemented data analysis and processing workflows for statistical evaluation (linear regression, Z-factor) in high-throughput screening of thousands of novel small-molecules using FACS Flow Cytometry techniques for measuring antiviral activities.}
    % \setlength{\itemindent}{0.5em}
    \item   \small{Executed data management and documentation for R\&D experiments in support to data integrity, reporting, visualization, and experimental design.}
    % \item   \small{Facilitated data handling and documentation for R\&D experiments with script automation and by authoring Standard Operating Procedures (SOPs).}
    
\end{itemize}

\medskip

%%%%%%%%%%%%%%%%%%%%%%%%%%%%%%% Projects %%%%%%%%%%%%%%%%%%%%%%%%%%%%%%%

\cvsection[page2sidebar]{Projects and Research}

%%%%%%%%%%%%%
%Project 1
%%%%%%%%%%%%%

\projectevent{PyTorch Deep-Learning Time-Series Forecasting Library (2021)}{}{}{}

\begin{itemize}
  \item \small Contributed in the development of an efficient and user-friendly open-source library, with state-of-the-art research and benchmark reports, for the PyTorch community.
    \item \small Integrated and tested research code for Seq2Seq RNN and DCRNN models, for up to spatio-temporal predictions. Models were tested against traffic data, counts and speed, from CalTrans State Highway network of sensors to evaluate COVID-19 traffic trends. 
\end{itemize}

\medskip


%%%%%%%%%%%%%
%Project 2
%%%%%%%%%%%%%

\projectevent{Drug Epidemic Tracking Through Social Media (2020)}{}{}{}

\begin{itemize}
  \item \small Built knowledge graph with Data Integration, NLP, and Entity-Matching strategies such as Blocking, Similarity Functions, Feature Vectors, and ML-Matchers.
    \item \small Consolidated a graph database schema (in Neo4j) to explore drug abuse and overdose patterns, with Cypher declarative queries, across the data ingested from Reddit posts and US News articles. 
\end{itemize}

\medskip

%%%%%%%%%%%%%
%Project 3
%%%%%%%%%%%%%


\projectevent{Predicting Helpfulness of Amazon Reviews (2020)}{}{}{}

\begin{itemize}
  \item \small Analyzed numerical and text data from 200,000 reviews in order to extract useful information with feature engineering and NLP techniques. Feature selection was done with an ablation study. 
    % \setlength{\itemindent}{0.5em}
    \item \small Developed a Random Forest Regressor Model to predict the ratio between the number of helpful votes with total votes, performance metrics used include MAE, MSE, Precision, and Recall. 
\end{itemize}

\medskip

%\divider

%%%%%%%%%%%%%
%Project 4
%%%%%%%%%%%%%

% \projectevent{Identifying Oncogene-Specific Essential Genes (2016)}{}{}{}

% \begin{itemize}
%   \item \small{Evaluated a bioinformatics prototype using R, based on distributional entropies and kernel-based density estimators to find drug sensitivity profiles that match patterns of gene expression with over 10,000 variables.}
% \end{itemize}


\end{document}
